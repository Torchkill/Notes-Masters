%====================================================
\section{Lecture 2}

\subsection{Modulation}

In practical wireless systems, information cannot be transmitted directly at baseband over long distances. Antennas are physically inefficient at very low frequencies, and multiple users cannot share the spectrum effectively without frequency separation. 

For this reason, a baseband message signal $m(t)$ is shifted to a higher frequency using a carrier wave of frequency $f_c$. This process is known as modulation.

The simplest modulation process multiplies the baseband signal by a cosine carrier:

\[
s(t) = m(t)\cos(2\pi f_c t).
\]

Here, $f_c$ is called the carrier frequency, and it determines where the transmitted signal appears in the frequency spectrum.

Modulation therefore performs two key roles:
\begin{enumerate}
    \item Enables practical radiation through antennas.
    \item Allows multiple signals to coexist at different carrier frequencies.
\end{enumerate}

%====================================================
\subsection{Baseband and Bandpass Signals}

A baseband signal is centered around 0 Hz in the frequency domain. Its spectrum extends from $-B$ to $B$, where $B$ is the bandwidth of the signal.

After modulation, the signal becomes a bandpass signal whose spectrum is centered at $f_c$.

\subsubsection*{Baseband Spectrum}

\begin{center}
\begin{tikzpicture}
\draw[->] (-4,0) -- (4,0) node[right] {Frequency};
\draw[->] (0,0) -- (0,2);
\draw[thick] (-1.5,0) -- (-1.5,1.5);
\draw[thick] (1.5,0) -- (1.5,1.5);
\node at (0,-0.6) {0 Hz};
\node at (0,-1.2) {Baseband};
\end{tikzpicture}
\end{center}

\subsubsection*{Bandpass Spectrum}

\begin{center}
\begin{tikzpicture}
\draw[->] (0,0) -- (10,0) node[right] {Frequency};
\draw[->] (0,0) -- (0,2);
\draw[thick] (3,0) -- (3,1.5);
\draw[thick] (7,0) -- (7,1.5);
\node at (5,-0.6) {$f_c$};
\node at (5,-1.2) {Bandpass};
\end{tikzpicture}
\end{center}

The bandwidth of the bandpass signal remains:

\[
B = f_H - f_L.
\]

A fundamental efficiency metric is spectral efficiency:

\[
\eta = \frac{R_b}{B}.
\]

This measures how efficiently bandwidth is used to transmit information.

%====================================================
\subsection{Amplitude Modulation (AM)}

In amplitude modulation, the amplitude of the carrier is varied proportionally to the message signal:

\[
s(t) = A_c [1 + k_a m(t)] \cos(2\pi f_c t).
\]

The quantity $k_a$ controls how strongly the message influences the carrier amplitude.

The modulation index is:

\[
\mu = k_a \max |m(t)|.
\]

If $\mu > 1$, the envelope crosses zero, causing overmodulation and distortion.

AM is simple to implement but inefficient in power because a large portion of transmitted energy resides in the carrier.

\subsubsection*{AM Envelope}

\begin{center}
\begin{tikzpicture}[scale=1]
\draw[->] (-0.5,0) -- (6,0) node[right] {$t$};
\draw[->] (0,-2) -- (0,2);
\draw[domain=0:6,smooth,variable=\x,blue]
plot ({\x},{(1+0.5*sin(2*pi*\x/4))*sin(2*pi*\x)});
\draw[domain=0:6,smooth,variable=\x,red,dashed]
plot ({\x},{(1+0.5*sin(2*pi*\x/4))});
\draw[domain=0:6,smooth,variable=\x,red,dashed]
plot ({\x},{-(1+0.5*sin(2*pi*\x/4))});
\end{tikzpicture}
\end{center}

%====================================================
\subsection{Double Sideband Suppressed Carrier (DSB-SC)}

If the carrier is removed and only the product $m(t)\cos(2\pi f_c t)$ is transmitted, the scheme is called DSB-SC.

\[
s(t) = m(t)\cos(2\pi f_c t).
\]

The resulting spectrum consists of two shifted copies of the baseband spectrum, producing bandwidth:

\[
B_{DSB} = 2B.
\]

This improves power efficiency but still occupies twice the baseband bandwidth.

%====================================================
\subsection{Frequency Modulation (FM)}

In frequency modulation, information is conveyed by varying the instantaneous frequency of the carrier:

\[
s(t) = A_c \cos\left(2\pi f_c t + 2\pi k_f \int m(t) dt \right).
\]

The peak frequency deviation is:

\[
\Delta f = k_f A_m.
\]

Unlike AM, FM has constant amplitude, making it more resistant to noise.

Practical bandwidth is estimated using Carson’s rule:

\[
B_{FM} = 2(\Delta f + B).
\]

This shows that FM bandwidth increases with deviation.

%====================================================
\subsection{Phase Modulation (PM)}

In phase modulation, the carrier phase varies directly with the message:

\[
s(t) = A_c \cos(2\pi f_c t + k_p m(t)).
\]

FM and PM are closely related. Both belong to the class of angle modulation schemes.

%====================================================
\subsection{Complex Baseband Representation}

Bandpass signals can be expressed using complex notation:

\[
s(t) = I(t) + jQ(t) = A e^{j\theta}.
\]

This representation simplifies analysis and forms the foundation of digital modulation techniques.

Symbols are transmitted over finite intervals:

\[
R_s = \frac{1}{T_s}.
\]

If $M$ symbols are used:

\[
R_b = R_s \log_2 M.
\]

Higher $M$ increases data rate but reduces noise tolerance.

%====================================================
\subsection{Multipath Propagation}

Wireless signals rarely follow a single direct path. Instead, reflections from buildings, ground, and objects create multiple delayed copies of the signal.

This leads to:
\begin{itemize}
\item Inter-Symbol Interference (ISI)
\item Fading
\end{itemize}

\begin{center}
\begin{tikzpicture}
\node (tx) at (0,0) {Tx};
\node (rx) at (8,0) {Rx};
\draw[->] (tx) -- (rx);
\draw[->] (tx) to[out=40,in=140] (rx);
\draw[->] (tx) to[out=-40,in=-140] (rx);
\end{tikzpicture}
\end{center}

%====================================================
\subsection{Free Space Propagation Model (Exam Important)}

In ideal free space:

\[
P_r = P_t \left(\frac{\lambda}{4\pi d}\right)^2.
\]

Since $\lambda = \frac{c}{f_c}$, higher carrier frequencies produce greater path loss.

Received power decays proportionally to:

\[
d^2.
\]

This quadratic decay is fundamental in link budget design.

%====================================================
\subsection{Duplexing}

Communication may occur in one or both directions:

Simplex allows one-way communication.

Half-duplex allows two-way communication, but not simultaneously.

Full-duplex allows simultaneous transmission and reception.

%====================================================
\subsection{Multiple Access Techniques}

To allow multiple users to share limited spectrum, several strategies are used.

FDMA separates users in frequency.

TDMA separates users in time.

CDMA separates users using orthogonal codes:

\[
s(t) = m(t)c(t).
\]

Modern systems use OFDM and OFDMA to dynamically allocate subcarriers.

%====================================================
\subsection{Path Loss Model}

Real environments are more complex than free space. A general empirical model is:

\[
PL(d) = PL(d_0) + 10n \log_{10}\left(\frac{d}{d_0}\right).
\]

The exponent $n$ depends on the environment.

%====================================================
\subsection{Channel Behavior}

Wireless channels may be:
Line-of-sight (LOS), where a direct path exists.
Time-invariant, where channel conditions remain stable.
Time-variant, where motion causes fading and Doppler effects.
Understanding channel behavior is essential for robust system design.
