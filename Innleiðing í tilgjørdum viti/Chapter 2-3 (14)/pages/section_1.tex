
A search problem is the foundational concept in AI problem-solving.
The agent must decide which actions to take to transition from the initial state to a goal state,
while considering the path cost and possible state transitions.

\section{Elements of a Search Problem (Common exam question)}

\begin{itemize}
    \item Initial State :  The starting point of the agent in the problem space.
    \item Actions : A set of all possible actions the agent can take from a state.
    \item Transition Model: A function $Result(s, a)$ that returns the resulting state from performing action $a$ in state $s$.
    \item States: All possible configurations or situations the agent may encounter.
    \item Goal Test: A function that checks if a given state is a goal state.
    \item Path Cost: A numerical value that defines the cost associated with a path from the initial state to a given state.
\end{itemize}

Key point: A solution is a sequence of actions that leads from the initial state to the goal state.

\subsection{Additional Key Concepts}

State space: The complete set of all possible states reachable from the initial state, given the actions and transition model.\\

Search Tree: A structure where nodes represent states, and branches represent actions leading to those states. \\

Search Graph (vs. Tree): Includes cycles and shared states — more efficient than trees but requires cycle detection.\\

Node (in a search tree/graph):  
  A data structure that typically contains:
  \begin{itemize}
    \item State  
    \item Parent Node
    \item Action taken to reach this state
    \item Path cost so far
    \item Depth (optional)
  \end{itemize}

Branching Factor (b): The average number of successors (child nodes) per state. \\

Belief State:  A representation of all the information an agent has about the world (useful in partially observable environments). Example: tracking visited locations.\\

Directed Acyclic Graph (DAG):  A graph with directed edges and no cycles.

\section{Search Strategies}

\subsection{Uninformed Search}

Uninformed (Blind) Search has no domain-specific knowledge is used.\\

Strategies:

\begin{itemize}
    \item Breadth-First Search (BFS)
    \item Depth-First Search (DFS)
    \item Uniform-Cost Search (Lowest-Cost-First)
    \item Iterative Deepening DFS\\
\end{itemize}

\subsection{Search Algorithm Structure}

Generic Graph Search\\

\begin{itemize}
    \item Initialize:\begin{itemize}
        \item Frontier = {Initial State}
        \item Explored set = empty set
    \end{itemize}
    \item Loop \begin{itemize}
        \item If Frontier is empty return no solution
        \item Remove node from Frontier
        \item If node.state is goal return the solution
        \item add node.state to Explored set
        \item Expand node and add resulting nodes to Frontier (Only if not in Frontier or Explored set)\\
    \end{itemize}
\end{itemize}

 Depth-First Search (DFS):
 \begin{itemize}
    \item Frontier: Stack (Last in first out)
    \item Explores the deepest node first.
    \item Pros: Low memory usage
    \item Cons: Can get stuck in deep or infinite paths
    \item Time complexity: $O(b^m)$
    \item Space complexity: $O(b \cdot m)$\\
 \end{itemize}

 Where $b$ is the branching factor and $m$ is the maximum depth.\\\\

Iterative Deepening DFS (IDDFS)
\begin{itemize}
    \item Combines benefits of BFS and DFS.
    \item Repeatedly runs DFS with increasing depth limits.
    \item Pros: Finds optimal solution with less space
    \item Redundant expansions\\
\end{itemize}


\textbf{Simplified Pseudocode:}
\begin{verbatim}
for depth = 0 to infinity:
    result = Depth-Limited-Search(initial, depth)
    if result not equal failure:
        return result
\end{verbatim}

Breadth-First Search (BFS)
\begin{itemize}
    \item Frontier: Queue (FIFO)
    \item Explores the shallowest nodes first.
    \item Guarantees: Finds the solution with the fewest steps (minimum depth)
    \item Time complexity: $O(b^d)$  
    \item Space complexity: $O(b^d)$\\
\end{itemize}

Where $b$ is the branching factor and $d$ is depth of the shallowest solution.\\\\


Uniform-Cost Search (UCS)
\begin{itemize}
    \item Expands the node with the lowest path cost.
    \item Uses a priority queue, ordered by path cost.
    \item Guarantees:Finds optimal solution (lowest cost)
    \item Time complexity: Depends on cost granularity (can be exponential)\\
\end{itemize}
-

Informed (Heuristic) Search
\begin{itemize}
    \item Uses domain-specific knowledge (heuristics) to guide search.
    \item Examples: $A*$ Search and Greedy Best-First Search\\
\end{itemize}


\section{Heuristic (Informed) Search}
Heuristic search leverages problem-specific knowledge to guide exploration more efficiently than uninformed strategies.

\subsection{Heuristic Function $h(n)$}
A heuristic is a non-negative estimate of the cost from node $n$ to the goal.
\begin{itemize}
    \item Example: Euclidean (straight-line) distance in a map.
\end{itemize}

\subsection{Admissibility}
A heuristic is \textbf{admissible} if it never overestimates the true cost:
\[
h(n) \leq \text{true cost from $n$ to goal}
\]
\begin{itemize}
    \item Examples: Euclidean distance, Manhattan distance (in grid environments without obstacles).
\end{itemize}

\subsection{Consistency (Monotonicity)}
A heuristic is \textbf{consistent} if:
\[
h(n) \leq c(n, n') + h(n')
\]
where $c(n, n')$ is the cost of moving from $n$ to $n'$.
\begin{itemize}
    \item Equivalent to the triangle inequality.
    \item Ensures the total estimated cost $f(n) = g(n) + h(n)$ is non-decreasing along a path.
\end{itemize}



\section{Summary of Key Terms}
\begin{itemize}
    \item Agent: Entity making decisions
    \item Environment: The world the agent operates in
    \item Action: Operation an agent can take
    \item Frontier: Set of nodes available for expansion
    \item Explored Set: Set of visited states
    \item Goal State: Target state for the agent
    \item Path Cost: Numerical cost of reaching a node
    \item Optimal Solution: The solution with the lowest total path cost
\end{itemize}

\textbf{Exam Tips:}
\begin{itemize}
    \item List and explain the 5 elements of a search problem
    \item Distinguish between search tree and search graph
    \item Compare time and space complexity for BFS, DFS, IDDFS
    \item Know pros/cons of different search methods
\end{itemize}
