\section{Advanced First-Order Logic and Optimization Techniques}

\subsection{Universal and Existential Quantification}
\begin{itemize}
    \item \textbf{Universal Quantification:} $\forall x \, P(x)$ — "for all $x$, $P(x)$ is true".
    \item \textbf{Existential Quantification:} $\exists x \, P(x)$ — "there exists an $x$ such that $P(x)$ is true".
    \item These quantifiers are fundamental in First-Order Logic (FOL) for expressing general statements about objects.
\end{itemize}

\subsection{First-Order Logic (FOL) Formulas}
\begin{itemize}
    \item \textbf{Arity:} Number of arguments a predicate or function takes.
    \item \textbf{Predicates with functions:} Allow expressing relationships between objects and their properties.
    \item \textbf{Interpretations:} Map constants, functions, and predicates to values or truth assignments.  
        Example: $F = c_1 + c_3 > c_2$, domain $\{1,2,3,4\}$  
        Relations are sets of tuples where the interpretation is true.
\end{itemize}

\subsection{Key FOL Techniques}
\begin{itemize}
    \item \textbf{Skolemization:} Remove existential quantifiers by introducing Skolem constants or functions.
    \item \textbf{Unification:} Process of finding substitutions for variables to make terms or predicates match.
    \item \textbf{Resolution in FOL:} Extends propositional resolution to FOL using unification to derive new clauses.
\end{itemize}

\subsection{Soundness and Completeness in FOL}
\begin{itemize}
    \item \textbf{Soundness:} Every derived formula is logically true.
    \item \textbf{Completeness:} Every logically entailed formula can be derived using the inference system.
\end{itemize}

\subsection{Limitations of First-Order Logic}
\begin{itemize}
    \item Incompleteness in some domains
    \item Scalability challenges with large knowledge bases
    \item Difficult to represent spatial and temporal reasoning
    \item Handling uncertainty is non-trivial
\end{itemize}

\subsection{Tautology}
\begin{itemize}
    \item A formula that is true under every possible interpretation.
    \item Important in both propositional logic and FOL.
\end{itemize}

\section{Optimization Concepts}

\subsection{Definition and Objective}
\begin{itemize}
    \item Optimization: Selecting the best option from a set of alternatives.
    \item Formalization:
    \begin{itemize}
        \item Variables with associated domains
        \item Objective function mapping assignments to real numbers
        \item Optimality criterion: find assignment minimizing or maximizing the objective function
    \end{itemize}
    \item Example: Minimizing loss associated with a variable assignment.
\end{itemize}

\subsection{Local Search Algorithms}
\begin{itemize}
    \item Maintain a single current state and explore neighboring states to improve the objective.
    \item \textbf{Example:} Placing houses and hospitals in 2D space to minimize total distance.
    \item \textbf{State space landscape:} Visualizes global/local maxima and minima, flat regions, and shoulders.
\end{itemize}

\subsection{Hill Climbing Variants}
\begin{itemize}
    \item \textbf{Steepest-Ascent Hill Climbing:} Move to the neighbor with the highest improvement.
        \begin{itemize}
            \item Issues: Local maxima, plateaus, ridges
        \end{itemize}
    \item \textbf{Stochastic Hill Climbing:} Randomly select among improving neighbors.
    \item \textbf{First-Choice Hill Climbing:} Randomly evaluate neighbors and move to first improvement found.
    \item \textbf{Random-Restart Hill Climbing:} Restart from random initial states to escape local maxima.
    \item \textbf{Local Beam Search:} Maintain $k$ states in parallel, keep best successors at each step.
\end{itemize}

\subsection{Iterative Improvement Techniques}
\begin{itemize}
    \item \textbf{Simulated Annealing:} Accept worse neighbors with decreasing probability over time to escape local optima.
    \item \textbf{Gradient Descent:} For continuous domains, adjust variables proportionally to reduce the objective function.
\end{itemize}

\subsection{Genetic Algorithms}
\begin{itemize}
    \item Maintain a population of candidate solutions.
    \item Randomly select pairs and perform crossover to produce offspring.
    \item Apply mutation to introduce variability.
    \item Iterate until an acceptable solution is found.
\end{itemize}

\subsection{Key Considerations}
\begin{itemize}
    \item Local search is fast but can get stuck in local optima.
    \item Random restarts, stochastic methods, and population-based methods improve robustness.
    \item Continuous vs discrete domains may require specialized algorithms (e.g., gradient descent for continuous).
\end{itemize}

