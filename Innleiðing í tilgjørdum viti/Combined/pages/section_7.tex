
\section{Introduction to AI Learning}

Artificial Intelligence (AI) learning refers to the process by which an AI system improves its performance over time based on data, experience, or feedback. This process can:

\begin{itemize}
    \item Expand the range of behaviors.
    \item Improve accuracy on tasks.
    \item Increase speed of task completion.
\end{itemize}

\subsection{Core Components of Learning}

\begin{itemize}
    \item \textbf{Task:} The specific behavior or function being improved.
    \item \textbf{Data:} The experiences (examples, feedback, etc.) used to improve performance.
    \item \textbf{Measure of Improvement:} The metric used to quantify learning progress.
\end{itemize}

\subsection{Large Data Requirements}

AI systems often require large datasets to generalize effectively. The learner operates as a black box that takes:

\begin{itemize}
    \item Experiences
    \item Problem description
    \item Background knowledge or bias
    \item And produces answers or decisions
\end{itemize}

\textbf{Learner} $\rightarrow$ \textbf{Models} $\rightarrow$ \textbf{Reasoner}

\subsection{Common Learning Paradigms}

\begin{itemize}
    \item Supervised Classification
    \item Unsupervised Learning
    \item Reinforcement Learning
    \item Analytic Learning
    \item Inductive Logic Programming
    \item Statistical Relational Learning
\end{itemize}

\subsection{Feedback and Bias}

\textbf{Feedback:} Determines how success is measured.

\begin{itemize}
    \item $P$: Assumes only observed negative examples are negative.
    \item $N$: Assumes only observed positive examples are positive.
\end{itemize}

\textbf{Bias:} Different assumptions about the distribution of examples, influencing the hypothesis space.

Example: A linear model or a polynomial fit may result from differing biases over the same data.

\subsection{Learning as Search}

Learning can be viewed as searching through a space of hypotheses or models.

\begin{itemize}
    \item The search space is often very large.
    \item Techniques like gradient descent and stochastic simulation are used.
\end{itemize}

\textbf{Learning algorithm} = Search Space + Evaluation Function + Search Method

\subsection{Challenges in Learning}

\textbf{Data Issues:}

\begin{itemize}
    \item Inadequate features
    \item Missing or noisy data
    \item Incorrectly labeled features
\end{itemize}

\textbf{Types of Error:}

\begin{itemize}
    \item Limited representation (representation bias)
    \item Limited search (search bias)
    \item Limited data (variance)
    \item Noisy features (noise)
\end{itemize}

\subsection{Model Representation and Bias-Variance Tradeoff}

\begin{itemize}
    \item Richer representations enable better problem-solving.
    \item But they are harder to learn and more prone to overfitting.
    \item Tradeoff exists between bias (underfitting) and variance (overfitting).
\end{itemize}

\section{Supervised Learning}

Given a dataset of input-output pairs, the goal is to learn a function $f$ that maps inputs to outputs.

\subsection{Terminology}

\begin{itemize}
    \item Input Features
    \item Target Features
    \item Training Examples
\end{itemize}

\subsection{Classification}

Learning a function to map input to a discrete category (e.g., spam or not spam). Target features $Y_i$ are discrete.

\textbf{Nearest Neighbour Classification:} Choose the class of the nearest training example.

\textbf{K-Nearest Neighbour Classification:} Choose class by majority vote among $k$ nearest examples.

\subsection{Linear Classification}

A linear decision boundary:

\[
x_1 = \text{humidity},\quad x_2 = \text{pressure}
\]
\[
h(x_1, x_2) =
\begin{cases}
\text{rain} & \text{if } w_0 + w_1x_1 + w_2x_2 \geq 0 \\
\text{no rain} & \text{otherwise}
\end{cases}
\]

\[
\text{Input Vector: } x = (1, x_1, x_2),\quad \text{Weight Vector: } w = (w_0, w_1, w_2)
\]
\[
h_w(x) =
\begin{cases}
1 & \text{if } w \cdot x \geq 0 \\
0 & \text{otherwise}
\end{cases}
\]

\textbf{Perceptron Learning Rule:}

\[
w_i \leftarrow w_i + \alpha (y - h_w(x)) x_i
\]

\subsection{Activation Functions}

\begin{itemize}
    \item \textbf{Step function:} Hard threshold.
    \item \textbf{Sigmoid function:} Soft threshold.
    \[
    f(x) = \frac{1}{1 + e^{-x}}
    \]
\end{itemize}

\textbf{Logistic Regression:} Minimize the error of the logistic function by adjusting weights.

\subsection{Support Vector Machines}

\begin{itemize}
    \item Adjusts the decision boundary to maximize the margin.
    \item \textbf{Linearly Separable:} Exists a hyperplane where classes are separable.
    \item Projects data into higher-dimensional space if necessary.
\end{itemize}

\section{Regression}

Supervised learning task to predict continuous values.

\[
\text{Goal: predict target } Y \text{ from inputs } X_1, X_2, \ldots, X_n
\]

\textbf{Linear Regression:}

\[
\text{Minimize Sum of Squared Errors (SSE):}
\quad SSE(E, w) = \sum_{e \in E} (o_e - p_e)^2
\]

\textbf{Stochastic Gradient Descent:} Update weights incrementally after each example.

\section{Evaluating Hypotheses}

\begin{itemize}
    \item $o_e$: Observed value
    \item $p_e$: Predicted value
    \item Error: $|o_e - p_e|$
\end{itemize}

\subsection{Loss Functions}

\begin{itemize}
    \item \textbf{0-1 Loss:}
    \[
    L(o, p) =
    \begin{cases}
    0 & \text{if } o = p \\
    1 & \text{otherwise}
    \end{cases}
    \]
    \item \textbf{$L_1$ Loss:} Absolute error
    \item \textbf{$L_2$ Loss:} Squared error
    \[
    L = (o - p)^2
    \]
    \item \textbf{$L_\infty$ Loss:} Max error
\end{itemize}

\section{Overfitting and Regularisation}

\textbf{Overfitting:} Model captures noise in training data, failing to generalize.

\textbf{Regularisation:}

\[
\text{Cost}(h) = \text{Loss}(h) + \lambda \cdot \text{Complexity}(h)
\]

\begin{itemize}
    \item Encourages simpler models.
    \item Feature selection: Removing irrelevant features.
\end{itemize}

\section{Model Evaluation Techniques}

\begin{itemize}
    \item Holdout Validation: Train/test split.
    \item K-Fold Cross Validation: Split into $k$ parts and test on each part iteratively.
\end{itemize}

\section{Practical Tools}

\textbf{Scikit-learn:} Python machine learning library offering tools for classification, regression, clustering, and evaluation.
