\section{Linear Algebra}
% Short description of chapter

\subsection{Matrix}
A matrix is usually written as $\mathbf{A} \in \mathbb{R}^{m \times n}$ where $m$ means the amount of rows and $n$ is the amount of columns. An example of a matrix:
$$
\mathbf{A} =
\begin{bmatrix}
    a_{11} & a_{12} & \cdots & a_{1n} \\
    a_{21} & a_{22} & \cdots & a_{2n} \\
    \vdots & \vdots & \ddots & \vdots \\
    a_{m1} & a_{m2} & \cdots & a_{mn}
\end{bmatrix},
\quad a_{ij} \in \mathbb{R}
$$

\subsection{Matrix addition}
\begin{center}
\begin{tabular}{|l|c|}
    \hline
    \textbf{Property} & \textbf{Description} \\ \hline
    Associative & \((A + B) + C = A + (B + C)\) \\ \hline
    Commutative & \(A + B = B + A\) \\ \hline
    Identity & \(A + O = O + A = A\), where \(O\) is the zero matrix \\ \hline
    Inverse & \(A + (-A) = O\), where \(-A\) is the additive inverse \\ \hline
\end{tabular}
\end{center}

\textbf{Example}
\begin{align*}
\end{align*}

\subsection{Matrix multiplication}
\begin{center}
\begin{tabular}{|l|c|}
    \hline
    \textbf{Property} & \textbf{Description} \\ \hline
    Associative & \((AB)C = A(BC)\) \\ \hline
    Distributive over Addition & \(A(B + C) = AB + AC\) \\
                               & \((A + B)C = AC + BC\) \\ \hline
    Identity & \(AI = IA = A\), where \(I\) is the identity matrix \\ \hline
    Non-commutative & \(AB \neq BA\) (in general) \\ \hline
\end{tabular}
\end{center}

Make sure to check the dimensions of the matrices, that they are compatible and that the resulting matrix has the correct dimensions.
On the example below $A \cdot B = C$ notice the dimensions.
\begin{align*}
  A \in& \mathbb{R}^{n \times m}\\
  B \in& \mathbb{R}^{m \times p}\\
  C \in& \mathbb{R}^{n \times p}\\
\end{align*}

\textbf{Example}
\begin{align*}
A =& \begin{bmatrix}
1 & 2 & 3 \\
3 & 2 & 1
\end{bmatrix} \in \mathbb{R}^{2 \times 3},\\
  B =& \begin{bmatrix}
0 & 2 \\
1 & -1 \\
0 & 1
\end{bmatrix} \in \mathbb{R}^{3 \times 2},\\
  AB =& \begin{bmatrix}
1 & 2 & 3 \\
3 & 2 & 1
\end{bmatrix}
\begin{bmatrix}
0 & 2 \\
1 & -1 \\
0 & 1
\end{bmatrix}
= \begin{bmatrix}
2 & 3 \\
2 & 5
\end{bmatrix} \in \mathbb{R}^{2 \times 2},\\
BA =& \begin{bmatrix}
0 & 2 \\
1 & -1 \\
0 & 1
\end{bmatrix}
\begin{bmatrix}
1 & 2 & 3 \\
3 & 2 & 1
\end{bmatrix}
= \begin{bmatrix}
6 & 4 & 2 \\
-2 & 0 & 2 \\
3 & 2 & 1
\end{bmatrix} \in \mathbb{R}^{3 \times 3}.
\end{align*}

\subsection{Inverse and determinant}
A matrix only has a inverse if its determinant is not 0 so always calculate it before finding the inverse. If it exists the following holds:
$$
\mathbf{AB} = \mathbf{BA} = \mathbf{I}_n
$$
\noindent
The determinant is easy to find for a $2 \times 2$ matrix:
$$
Det(\mathbf{A}) =
\begin{bmatrix}
  a_{11} & a_{12}\\
  a_{21} & a_{22}
\end{bmatrix}
= a_{11}a_{22} - a_{21}a{12}
$$

After checking that the determinant is not zero proceed to find the inverse.
If you're working with a $2 \times 2$ matrix the following setup gives the inverse, fig. \ref{2x2_matrix_inverse}

%\begin{align*}
%
%\end{align*}


\noindent
For a $3 \times 3$ matrix Sarrus's rule can be used.\\
For matrices $3 \times 3$ and larger one can use Laplace expansion.


\subsection{Transposed matrix}
To find the transpose 
$$
  \mathbf{A} = \begin{bmatrix}
a & b & c \\
d & e & f
\end{bmatrix}
$$
then
$$
  \mathbf{A}^T =
  \begin{bmatrix}
a & d \\
b & e \\
c & f
\end{bmatrix}
$$

%\subsection{Multiplication with scalar}

\subsection{Particular and general solution}

\subsection{Gauss elimination}
When doing gaussian elimination you are allowed to do the following operations:
\begin{itemize}
  \item Exchange of two equations (rows in the matrix representing the system of equations)
  \item Multiplication of an equation (row) with a constant $\lambda \in \mathbb{R}\{0\}$
  \item Addition of two equations (rows)
\end{itemize}
\noindent
\textbf{Definition 2.6} (Row-Echelon Form). A matrix is in row-echelon form if:
\begin{itemize}
  \item All rows that contain only zeros are at the bottom of the matrix; correspondingly, all rows that contain at least one nonzero element are on top of rows that contain only zeros.
  \item Looking at nonzero rows only, the first nonzero number from the left (also called the pivot or the leading coefficient) is always strictly to the right of the pivot of the row above it.
\end{itemize}

\noindent
\textbf{Remark (Reduced Row Echelon Form).}
An equation system is in \textit{reduced row-echelon form} (also: \textit{row-reduced echelon form} or \textit{row canonical form}) if
\begin{itemize}
    \item It is in row-echelon form.
    \item Every pivot is 1.
    \item The pivot is the only nonzero entry in its column.
\end{itemize}

\textbf{Minus-1 trick}

\subsection{Groups}
\textbf{Definition 2.7 (Group).}
Consider a set \( G \) and an operation \( \otimes: G \times G \to G \) defined on \( G \).
Then \( G := (G, \otimes) \) is called a \textit{group} if the following hold:

\begin{enumerate}
    \item \textbf{Closure} of \( G \) under \( \otimes \): \( \forall x, y \in G: x \otimes y \in G \)

    \item \textbf{Associativity}: \( \forall x, y, z \in G: (x \otimes y) \otimes z = x \otimes (y \otimes z) \)

    \item \textbf{Neutral element}: \( \exists e \in G \, \forall x \in G: x \otimes e = x \) and \( e \otimes x = x \)

    \item \textbf{Inverse element}: \( \forall x \in G \, \exists y \in G: x \otimes y = e \) and \( y \otimes x = e \), where \( e \) is the neutral element. We often write \( x^{-1} \) to denote the inverse element of \( x \).
\end{enumerate}

\textit{Remark} The inverse element is defined with respect to the operation \( \otimes \) and does not necessarily mean \( \frac{1}{x} \). \(\diamondsuit$

\subsection{Vector spaces}
\begin{definition}[Vector Space]
A real-valued \emph{vector space} \( V = (\mathcal{V}, +, \cdot) \) is a set \(\mathcal{V}\) with two operations
\[
+ : \mathcal{V} \times \mathcal{V} \to \mathcal{V}
\tag{2.62}
\]
\[
\cdot : \mathbb{R} \times \mathcal{V} \to \mathcal{V}
\tag{2.63}
\]
where
\begin{enumerate}
    \item \((\mathcal{V}, +)\) is an Abelian group.
    \item Distributivity:
    \begin{enumerate}
        \item \(\forall \lambda \in \mathbb{R},\ x,y \in \mathcal{V}: 
        \lambda \cdot (x + y) = \lambda \cdot x + \lambda \cdot y\).
        \item \(\forall \lambda, \psi \in \mathbb{R},\ x \in \mathcal{V}:
        (\lambda + \psi) \cdot x = \lambda \cdot x + \psi \cdot x\).
    \end{enumerate}
    \item Associativity (outer operation): 
    \(\forall \lambda, \psi \in \mathbb{R},\ x \in \mathcal{V}:
    \lambda \cdot (\psi \cdot x) = (\lambda \psi) \cdot x\).
    \item Neutral element with respect to the outer operation:
    \(\forall x \in \mathcal{V}:\ 1 \cdot x = x\).
\end{enumerate}
\end{definition}

\subsection{Basis and rank}
To determine a basis take the vectors that span the vector subspace and put them into matrix form:

\[
  \vb x_1 =
  \begin{bmatrix}
    1\\2\\-1\\-1\\-1
  \end{bmatrix}, \quad
  \vb x_2 =
  \begin{bmatrix}
    2\\-1\\1\\2\\-2
  \end{bmatrix}, \quad
  \vb x_3 =
  \begin{bmatrix}
    3\\-4\\3\\5\\-3
  \end{bmatrix}, \quad
  \vb x_4 =
  \begin{bmatrix}
    -1\\8\\-5\\-7\\1
  \end{bmatrix}
\]

We check that the following holds:
\[
  \sum_{i=1}^4 \lambda_i \vb x_i = \vb 0
\]

\begin{align*}
  \left[ \vb x_1, \vb x_2, \vb x_3, \vb x_4\right] =&
  \begin{bmatrix}
    1 & 2 & 3 & -1\\
    2 & -1 & -4 & 8\\
    -1 & 1 & 3 & -5\\
    -1 & 2 & 5 & -6\\
    -1 & -2 & -3 & 1
  \end{bmatrix}\\
  \rightsquigarrow&
  \begin{bmatrix}
    1 & 2 & 3 & -3\\
    \cmidrule{0-1}
    0 & 1 & 2 & -2\\
    \cmidrule{1-3}
    0 & 0 & 0 & 1\\
    \cmidrule{3-4}
    0 & 0 & 0 & 0\\
    0 & 0 & 0 & 0
  \end{bmatrix}
\end{align*}
The resulting matrix has three columns with pivots. Those form a basis for the subspace spanned by those four vectors.\\
%The resulting matrix has three columns with pivots thus it is not full rank, as that would be a pivot on each column.\\
We see that there are two dependent rows, those that are all zeros and that there is one free variable $\vb x_3$ hence the rank is 3:
\[
  rk(A) = 3
\]
The column with no pivot, $\vb x_3$, 

A matrix is full rank if the rank equals the largest possible rank for a matrix of the same dimensions.

\subsection{Linear mappings}
A \textbf{linear mapping} (or linear transformation) \(\Phi: V \to W\) between vector spaces \(V\) and \(W\) over the same field preserves vector addition and scalar multiplication. For all \(\mathbf{x}, \mathbf{y} \in V\) and \(\lambda, \psi \in \mathbb{R}\):
\[
\Phi(\mathbf{x} + \mathbf{y}) = \Phi(\mathbf{x}) + \Phi(\mathbf{y}), \quad \Phi(\lambda \mathbf{x}) = \lambda \Phi(\mathbf{x}).
\]
This can be summarized as:
\[
\Phi(\lambda \mathbf{x} + \psi \mathbf{y}) = \lambda \Phi(\mathbf{x}) + \psi \Phi(\mathbf{y}).
\]

\subsection{Special Cases of Linear Mappings}
\begin{itemize}
    \item \textbf{Isomorphism}: A linear mapping \(\Phi: V \to W\) that is both injective and surjective (bijective). Two vector spaces are isomorphic if and only if \(\dim(V) = \dim(W)\).

    \item \textbf{Endomorphism}: A linear mapping \(\Phi: V \to V\) from a vector space to itself.

    \item \textbf{Automorphism}: An endomorphism that is also bijective.

    \item \textbf{Identity Mapping}: \(\text{id}_V: V \to V, \mathbf{x} \mapsto \mathbf{x}\).
\end{itemize}

\subsection{Injective, Surjective, and Bijective Mappings}
\begin{itemize}
    \item \textbf{Injective (One-to-One)}: \(\Phi(\mathbf{x}) = \Phi(\mathbf{y}) \implies \mathbf{x} = \mathbf{y}\).
    \item \textbf{Surjective (Onto)}: \(\Phi(V) = W\).
    \item \textbf{Bijective}: Both injective and surjective. A bijective mapping \(\Phi\) has an inverse \(\Phi^{-1}\).
\end{itemize}

\subsection{Matrix Representation of Linear Mappings}
For finite-dimensional vector spaces \(V\) and \(W\) with ordered bases \(B = (\mathbf{b}_1, \ldots, \mathbf{b}_n)\) and \(C = (\mathbf{c}_1, \ldots, \mathbf{c}_m)\), the \textbf{transformation matrix} \(\mathbf{A}_\Phi\) of \(\Phi\) is defined by:
\[
\Phi(\mathbf{b}_j) = \sum_{i=1}^m \alpha_{ij} \mathbf{c}_i, \quad \text{where } \mathbf{A}_\Phi(i,j) = \alpha_{ij}.
\]
If \(\hat{\mathbf{x}}\) and \(\hat{\mathbf{y}}\) are the coordinate vectors of \(\mathbf{x} \in V\) and \(\Phi(\mathbf{x}) \in W\) with respect to \(B\) and \(C\), then:
\[
\hat{\mathbf{y}} = \mathbf{A}_\Phi \hat{\mathbf{x}}.
\]

\subsection{Example: Transformation Matrix}
For a linear mapping \(\Phi: V \to W\) with bases \(B = (\mathbf{b}_1, \mathbf{b}_2, \mathbf{b}_3)\) and \(C = (\mathbf{c}_1, \mathbf{c}_2, \mathbf{c}_3, \mathbf{c}_4)\), if:
\[
\Phi(\mathbf{b}_1) = \mathbf{c}_1 - \mathbf{c}_2 + 3\mathbf{c}_3 - \mathbf{c}_4, \quad \Phi(\mathbf{b}_2) = 2\mathbf{c}_1 + \mathbf{c}_2 + 7\mathbf{c}_3 + 2\mathbf{c}_4, \quad \Phi(\mathbf{b}_3) = 3\mathbf{c}_2 + \mathbf{c}_3 + 4\mathbf{c}_4,
\]
the transformation matrix \(\mathbf{A}_\Phi\) is:
\[
\mathbf{A}_\Phi = \begin{bmatrix}
1 & 2 & 0 \\
-1 & 1 & 3 \\
3 & 7 & 1 \\
-1 & 2 & 4
\end{bmatrix}.
\]

\subsection{Basis Change}
If the bases in \(V\) and \(W\) are changed to \(\tilde{B}\) and \(\tilde{C}\), the new transformation matrix \(\tilde{\mathbf{A}}_\Phi\) is related to the original \(\mathbf{A}_\Phi\) by:
\[
\tilde{\mathbf{A}}_\Phi = \mathbf{T}^{-1} \mathbf{A}_\Phi \mathbf{S},
\]
where \(\mathbf{S}\) maps coordinates from \(\tilde{B}\) to \(B\), and \(\mathbf{T}\) maps coordinates from \(\tilde{C}\) to \(C\).

\subsection{Example: Basis Change}
For a linear mapping \(\Phi: \mathbb{R}^3 \to \mathbb{R}^4\) with standard bases \(B\) and \(C\), and new bases \(\tilde{B}\) and \(\tilde{C}\), the transformation matrices \(\mathbf{S}\) and \(\mathbf{T}\) are constructed by expressing the new basis vectors in terms of the old basis vectors. The new transformation matrix is:
\[
\tilde{\mathbf{A}}_\Phi = \mathbf{T}^{-1} \mathbf{A}_\Phi \mathbf{S}.
\]

\subsection{Image and Kernel}
For a linear mapping \(\Phi: V \to W\):
\begin{itemize}
    \item \textbf{Kernel (Null Space)}: \(\ker(\Phi) = \{\mathbf{v} \in V \mid \Phi(\mathbf{v}) = \mathbf{0}_W\}\). The kernel is a subspace of \(V\).

    \item \textbf{Image (Range)}: \(\text{Im}(\Phi) = \{\Phi(\mathbf{v}) \mid \mathbf{v} \in V\}\). The image is a subspace of \(W\).
\end{itemize}

\subsection{Rank-Nullity Theorem}
The \textbf{Rank-Nullity Theorem} states:
\[
\dim(\ker(\Phi)) + \dim(\text{Im}(\Phi)) = \dim(V).
\]
This theorem is fundamental in linear algebra and has several important consequences:
\begin{itemize}
    \item If \(\dim(\text{Im}(\Phi)) < \dim(V)\), then \(\ker(\Phi)\) is non-trivial.
    \item If \(\dim(V) = \dim(W)\), then \(\Phi\) is injective if and only if it is surjective if and only if it is bijective.
\end{itemize}

\subsection{Example: Image and Kernel}
For a linear mapping \(\Phi: \mathbb{R}^4 \to \mathbb{R}^2\) defined by:
\[
\Phi(\mathbf{x}) = \begin{bmatrix} 1 & 2 & -1 & 0 \\ 1 & 0 & 0 & 1 \end{bmatrix} \mathbf{x},
\]
the image is the span of the columns of the transformation matrix:
\[
\text{Im}(\Phi) = \text{span}\left\{\begin{bmatrix} 1 \\ 1 \end{bmatrix}, \begin{bmatrix} 2 \\ 0 \end{bmatrix}, \begin{bmatrix} -1 \\ 0 \end{bmatrix}, \begin{bmatrix} 0 \\ 1 \end{bmatrix}\right\}.
\]
The kernel is found by solving \(\mathbf{A}\mathbf{x} = \mathbf{0}\):
\[
\ker(\Phi) = \text{span}\left\{\begin{bmatrix} 0 \\ \frac{1}{2} \\ 1 \\ 0 \end{bmatrix}, \begin{bmatrix} -1 \\ \frac{1}{2} \\ 0 \\ 1 \end{bmatrix}\right\}.
\]

\subsection{Key Theorems and Results}
\begin{itemize}
    \item Two finite-dimensional vector spaces \(V\) and \(W\) are isomorphic if and only if \(\dim(V) = \dim(W)\).

    \item The composition of linear mappings is linear. If \(\Phi: V \to W\) and \(\Psi: W \to X\) are linear, then \(\Psi \circ \Phi: V \to X\) is also linear.

    \item If \(\Phi: V \to W\) is an isomorphism, then \(\Phi^{-1}: W \to V\) is also an isomorphism.
\end{itemize}

\subsection{Examples of Linear Mappings}
\begin{itemize}
    \item The mapping \(\Phi: \mathbb{R}^2 \to \mathbb{C}, \Phi(\mathbf{x}) = x_1 + i x_2\) is a homomorphism, justifying the representation of complex numbers as tuples in \(\mathbb{R}^2\).

    \item Linear transformations can represent geometric operations such as rotations, stretches, and reflections in \(\mathbb{R}^2\).
\end{itemize}

\subsection{Example: Geometric Transformations}
For a rotation by \(45^\circ\) in \(\mathbb{R}^2\), the transformation matrix is:
\[
\mathbf{A}_1 = \begin{bmatrix} \cos(\pi/4) & -\sin(\pi/4) \\ \sin(\pi/4) & \cos(\pi/4) \end{bmatrix}.
\]
For a stretch along the horizontal axis by a factor of 2, the transformation matrix is:
\[
\mathbf{A}_2 = \begin{bmatrix} 2 & 0 \\ 0 & 1 \end{bmatrix}.
\]

% 54 til 

\subsection{Affine spaces}
